\documentclass[11pt]{letter}
\usepackage[utf8]{inputenc}
\usepackage[english]{babel}
\usepackage{geometry}
\geometry{letterpaper, margin=1in}

\address{
    Thierry Warin \\
    HEC Montreal \\
    CIRANO \\
    Montreal, QC, Canada \\
    \texttt{thierry.warin@hec.ca}
}

\date{\today}

\begin{document}

\begin{letter}{
    Editor-in-Chief \\
    Journal of Transport Geography \\
    Elsevier
}

\opening{Dear Editor-in-Chief,}

We are pleased to submit our manuscript titled ``Temporal Graph Modeling for Sparse Maritime Networks'' for consideration for publication in the \textit{Journal of Transport Geography}.

This paper addresses the geographical dimensions of maritime transport networks, specifically examining the spatial dynamics of vessel movements and port connectivity in less active yet strategically important corridors like the Great Lakes--St. Lawrence (GLSL) system. Understanding the spatial patterns and temporal evolution of these maritime networks is critical for regional economic integration, trade flows, and infrastructure planning in both urban and rural coastal communities.

Our work strongly aligns with the aim and scope of the \textit{Journal of Transport Geography}, as it examines the spatial dynamics of maritime and intermodal transport networks, the linkages between port infrastructure nodes and their regional environments, and the methodological developments for analyzing transport geography. We utilize high-resolution Automatic Identification System (AIS) data to construct a spatiotemporal representation of the port network, revealing the geographical patterns of maritime mobility and connectivity.

Key contributions of our study include:
\begin{enumerate}
    \item A geographical methodology to address data sparsity in maritime networks through spatial node aggregation and temporal network decomposition, enabling better understanding of connectivity patterns in regions with intermittent traffic.
    \item An empirical analysis demonstrating how Temporal Graph Neural Networks (TGNNs) can capture the spatial and temporal dynamics of maritime transport, with the GAT-GRU architecture proving particularly effective for sparse regional networks.
    \item A counterfactual analysis framework that enables assessment of how exogenous shocks propagate through maritime networks, supporting regional resilience planning and understanding the spatial implications of disruptions to port operations and trade flows.
\end{enumerate}

We believe this research offers valuable insights for the journal's interdisciplinary audience of transport geographers, regional planners, and policy analysts interested in the spatial dimensions of maritime transport, the role of logistics networks in regional development, and methodological advances in analyzing transport geography using geo-spatial methods and digital data.

This manuscript describes original work and is not under consideration by any other journal. All authors have approved the manuscript and agree with its submission.

Thank you for receiving our manuscript and considering it for review. We appreciate your time and look forward to your response.

\closing{Sincerely,}

Thierry Warin, PhD \\
Professor, HEC Montreal \\
Fellow, CIRANO

\end{letter}
\end{document}